\documentclass[uplatex,dvipdfmx]{jsarticle}

% パッケージをインストールする
\usepackage{amsmath,amssymb,bm}
\usepackage{hoge}
\usepackage[dvipdfmx]{graphicx}
\usepackage{listings,jlisting} %日本語のコメントアウトをする場合jlistingが必要
\usepackage{comment}
% \usepackage{caption}


% ソースコードに関する設定
\usepackage{color}

\lstset{
  % デフォルトでC言語:optionで変更する
  language={C},
  % キーワードの色を変更する
  keywordstyle={\bfseries\color[rgb]{0,0,1}},
  backgroundcolor={\color[gray]{.95}},
  basicstyle={\ttfamily},
  identifierstyle={\small},
  commentstyle={\smallitshape},
  ndkeywordstyle={\small},
  stringstyle={\small\ttfamily},
  frame=single,
  breaklines=true,
  columns=[l]{fullflexible},
  numbers=left,
  xrightmargin=0zw,
  xleftmargin=3zw,
  numberstyle={\scriptsize},
  stepnumber=1,
  numbersep=1zw,
  lineskip=-0.5ex
}

% \code[lang]{caption}{label}{path}
\newcommand{\code}[3][C++]{\lstinputlisting[language={#1},caption={#2},label={#3}]}




\title{テンプレート}
\author{学籍番号 0000000 名前}
\date{\today}


\begin{document}
\maketitle

\section{TeXのテンプレート}
適当にレポート書く時に使います.

\section{環境}
\begin{itemize}
    \item macOS
    \item upLaTeX
    \item dvippdfmcx
    \item macOS
\end{itemize}

\section{最後に}
使いやすいように勝手に変更されていきます.

\section{対応したいこと}
\begin{itemize}
    \item 画像の挿入
    \item 引用を楽にする
    \item ソースコードをいい感じにする
\end{itemize}

\section{ソースコード}

\code[Java]{main}{huga}{src/Main.java}

\section{画像の挿入}

\img[]{img/neko.png}{wanwan}

\end{document}
